\section{Parallélisation}

Notre code est intrinsèquement divisé en deux étape, la première étant l'initialisation du
dictionnaire, la seconde est le calcul matricielle qui permet de converger vers un résultat.

Au delà du fait que l'initialisation est la partie qui prend le moins de temps à l'exécution,
phénomène d'autant plus remarquable si la taille des documents et leur nombre augmente, nous
avons conclu qu'il était compliqué de répartir cette initialisation sur plusieurs nœuds MPI sans
engendrer une graphe de communication très chargé. En effet les identifiants associés à un
document ou à un mot sont créés si ce mot/documents est relevé pour la première fois dans
l'initialisation. Par exemple on aura alors le cas où deux nœuds découvrent à des instants
différents dans leur processus de création de dictionnaire un document. Ce dit-document aura donc
un identifiant i sur le premier nœud et un identifiant j dans le second or nous devons avoir un
identifiant unique pour un élément du dictionnaire.

Nous nous sommes donc concentré sur la parallélisation du calcul matricielle. Une fois le
dictionnaire et son indexation inversée établis nous savons que les matrices de fréquence qui les
représentent ne serons plus modifiés au cours de l'algorithme. Ainsi nous avons décidé de
distribuer à chacun des sites ces deux matrices (Docs et Words). Notre algorithme consiste à
chaque étape à comparer les documents/mots deux à deux. Afin que la charge de travail entre
processus MPI soit équilibrée chaque processus devra effectuer

\[ \frac{N_d \times (N_d+1)}{P \times 2}\mbox{, et} \]
\[ Nw*(Nw+1)/(P*2) \]
comparaisons à chaque étape (avec $N_d$ = nombre total de documents et $N_w$ = nombre
total de mots).

Une étape est de la forme suivant :

\begin{verbatim}
Matrix_Docs  = dist_polia (Docs, Matrix_Words)
Matrix_Words = dist_polia (Words, Matrix_Docs)
\end{verbatim}

Les processus savent donc sur qu'elle plage de donnée ils effectuent leurs calculs. Une
étape consiste à recalculer les matrices de distances Matrix\_Docs et Matrix\_Words. Ainsi entre
deux appels de la fonction dist\_polia() chaque processus doit recevoir les nouvelles distances que
les processus restant viennent de calculer.
