\section{Conclusion}

La première conclusion qui peut venir à l'esprit est que
l'on n'a pas réussi à construire de champ lexical \----
on n'a même pas eu d'information sur comment analyser
les matrices de distance construites.

Ça a pourtant été une très bonne occasion de
mettre en place un cahier des charges très peu complet
avec une liberté de conception, et de transformer un
problème d'analyse de fichier en un problème d'analyse numérique
qu'on peut ramener à des situations précédemment résolues.

On a été déçu d'avoir été guidé uniquement à implémenter
une méthode de construction de distances sans explication,
sans conception et sans indication quant à la manière
d'analyser les résultats, alors que c'était le coeur
de l'implémentation. Il est bien plus difficile de réorganiser
une implémentation pour des performances / du parallélisme
lorsqu'on doit deviner quelle méthode elle représente.

Il était instructif de découvrir ce projet avec une personne
extérieure à la formation et de se rendre compte des besoins
d'une entreprise.

C'était aussi l'occasion de découvrir un vrai espace
de travail dans lequel prendre sa place pour tenter de
réaliser de bonnes performances
(nous remercions l'organisation du Master de nous avoir donné
accès à ce lieu).
Nous avions peu de tests à effectuer en vraie grandeur
donc nous avons étudié la scalabilité d'un programme
de simulation optimisé en TP.
C'est dommage car c'était une de nos seules occasions
de lancer un job qui pouvait vraiment passer à l'échelle.
