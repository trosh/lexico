\section{Etat de l'art}

\subsection{Parallélisation de la recherche de similarité entre
séquences protéiques sur GPU}

Une approche pour améliorer la recherche de similarités
entre protéiques a été mise en place.
Celle-ci est un exemple de ce qui a déjà été développé
en terme de parallélisation dans le domaine de
comparaison de mots.
La méthode proposée, notamment par les chercheurs
Van Hoa Nguyen et Dominique Lavenier,
se base sur une indexation complète des données en mémoire.
Cette démarche est un moyen pour exhiber un parallélisme
extrême sur des traitements simples.
Chacun de ces traitements sont indépendants et s'adaptent
au calcul sur GPU, Graphical Processing Units.

Les programmes en question sont:

\begin{itemize}
\item iBlASTP
\item iBLASTN
\end{itemize}

Ces derniers ont été exécutés sur une carte
NVIDIA GeForce 8800 GTX.
Le travail effectué par ces deux codes sont, d'une part,
la recherche de similarités entre deux blocs protéiques
ou entre un bloc protéique et un génome complet.
Les valeurs d'accélération obtenues s'échelonnent
de 5 à 10 par rapport aux deux codes de référence
dans le domaine, à savoir:

\begin{itemize}
\item BLASTP
\item TBLASTN
\end{itemize}

\subsubsection{Généralités}

Dans le domaine de la bioinformatique, la recherche de
similarités entre séquences génomiques est un point
essentiel. L'objectif de cette recherche est de repérer
des régions similaires dans des séquences d'ADN ou des
séquences protéiques. Par exemple, la requête d'un bloc
ayant un gène dont la nature est inconnue nécessite
d'effectuer ce type de recherche de similarités.
De façon générale, les résultats obtenus sont
fondamentalement des segments similaires présentant
une caractéristique identique élevé. Ainsi,
l'information attendu doit présenter deux banques
de séquence où sont précisément indiqués les couples
de nucléotides (pour l'ADN) ou les couples d'acides
aminés (pour les protéines).

Le développement des biotechnologies génère des
données génomiques de plus en plus volumineuses.
Nous pouvons citer, par exemple, le bloc d'aden GenBank
qui possède plus de 180 milliards de nucléotides et sa
taille est multipliée par un facteur entre 1.4 et 1.5
annuellement. Par conséquent, on constate le traitement
de ces grandes quantités de données requiert davantages
de temps, malgré l'amélioration des puissances des
ordinateurs. En effet, d'après les calculs effectués
la quantité de données en termes de nombre de
nucléotides augmente plus rapidement que la puissance
des machines exprimée en MIPS.
Ainsi, la requête d'alignement entre séquence génomique
est avant tout une requête par le contenu sur des
données en vrac. Pour réaliser cette recherche, il est
nécessaire de parcourir régulièrement l'ensemble
des blocs, et ce de la première séquence jusqu'à la
dernière séquence. De nombreux algorithmes pour faire
ressortir des alignements ont été proposés. Parmis ces
différents algrithmes, nous pouvons citer le pionnier
Smith-Waterman qui en a réalisé un en 1981.
L'algorithme de Smith-Waterman utilise des méthodes
de programmation dynamique et ont une complexité
quadratique. Les autres algorithmes élaborés,
principalement à partirde 1990, notamment le
programme BLAST, reposent sur une méthode heuristique
qui permet de localiser de petites banques identiques
potentiellement pertinentes. Ces algorithmes, beaucoup
plus efficace par rapport aux premiers, ont
principalement été élaborés pour obtenir une requête
performante dans les blocs génomiques. En revanche,
leur utilisation dans le cadre d'une comparaison
poussée n'est pas vraiment appropriée, car elle donne
des temps d'exécution non optimisés.
Cependant, pour parvenir à acclérer la recherche
de ces blocs de séquence, plusieurs possibilités nous sont
données, à savoir

\begin{itemize}
\item l'indexation des données en mémoire principale;
\item la parallélisation à gros grain sur des
machines de type cluster;
\item la parallélisation à grain fin sur des
structures spécialisées à base de VLSI/FPGA;
\item la parallélisation sur des GPUs
\end{itemize}

\subsubsection{Algorithme de recherche de similarités}

Le coeur de l'idée se base sur une observation: la
plupart des alignements significatifs ont une région
de forte ressemblance se caractérisant par une suite
de W caractères similaires entre les banques des deux
séquences qui représentent cet alignement.
Les zones dans lesquelles ces mots apparaissent sont alors
repérées. Dès lors qu'un mot est trouvé sur deux
portions de séquences distinctes, il est utilisé
comme point d'ancrage pour déterminer un alignement
plus important en essayant d'élargir, de part et
d'autre, le nombre d'appariement (match) entre symboles
similaires. Si l'élargissement amène à une forte
similarité (déterminée en nombre de match/mismatch),
sera efectué alors une dernière étape dont le
principe est d'étendre encore plus l'alignement en
tenant compte de l'inclusion ou de l'omission de
symboles (gap).

L'exemple ci-dessous illustre comment l'alignement
entre deux séquences protéiques est construit.
Ainsi, le mot de 3 caractères ARV, qui apparaisse
dans les deux séquences, est utilisé comme point
d'ancrage. I s'agit alors de l'étape 1. A partir
de ce point d'ancrage, les développeurs ont élargit
de part et d'autres tant que le ration macth/mismatch
est intéressant.
Ceci correspond à l'étape 2.
Enfin, l'équipe d'informatique a élargit de
nouveau l'alignement en tenant en compte les erreurs
de gap. Cette phase est l'étape 3.

{\bf Etape 1}

\begin{verbatim}
K V I T A R V T G S A Q W C D N
        | | |
K L I S A R V K G S Q F C T N P
\end{verbatim}

{\bf Étape 2}

\begin{verbatim}
K V I T A R V T G S A Q W C D N
|   |   | | |   | |
K L I S A R V K G S Q F C T N P
\end{verbatim}

{\bf Etape 3}

\begin{verbatim}
K V I T A R V T G S A Q W C D N
|       | | |   | |       |   |
K L I S A R V K G S - Q F C T N P
\end{verbatim}

Donc la recherche de similarité est une démarche qui
s'effectue en 3 étapes:

\begin{itemize}
\item la recherche d'un point d'ancrage (étape 1)
\item l'extension sans gap (étape 2)
\item l'extension avec gap (étape 3)
\end{itemize}

Le programme BLAST implémente cette heuristique.

\subsubsection{L'algorithme de BLASTP}

L'algorithme de BLASTP prend en entrée un bloc protéique
et une ou plusieurs banques de requête. L'algorithme
s'effectue en 5 étapes:

{\bf Algorithme de BLASTP}

\begin{verbatim}
1: indexation de la requête sur la base  - étape 1
       de mots de W caractères
2: pour tous les mots M de la banque
3:   si M appartient à la requête        - étape 2
4:     extension sans gap                - étape 3
5:   Si score >= S1
6:      extension avec gap               - étape 4
7:   Si score >= S2
8:      afficher l'alignement            - étape 5
\end{verbatim}

{\bf Etape 1:} indexation.
'équipe effectue un découpage de chaque
requête en mots chevauchant de taille W. Ces mots sont
alors stockés au sein d'une table de hachage

{\bf Etape 2:} requête de point d'ancrage. La banque de
séquence est lue séquentiellement, et ce du début
jusqu'à la fin. Cette lecture prend en compte chaque
mot chevauchant de W caractères.
Pour chaque mot, l'équipe de développement
recherche s'il apparait dans la table de hachage.
Une fois qu'un mot existe, cela signifie que l'on est
en présence d'un point d'ancrage.

{\bf Etape 3:} extension sans gap. A partir du point d'ancrage
les programmeurs ont cherché à élargir de part et d'autres afin
d'entrainer une extension sans gap. De façon précise,
BLASTP élabore un élargissement sans gap seulement dans
le cas où deux points d'ancrage sont proches l'un de
l'autre. De manière conventionnelle, la distance séparant
les deux points d'ancrage doit être inférieure à 40 acides
aminés. Un alignement sans gap qui inclue ces 2 points
d'ancrage est alors calculé. Si le score coorespondant
à cet alignement est supérieure à une valeur seuil S1.

{\bf Etape 4:} extension avec gap. au cours de cette phase,
les déveoppeurs ont cherché de nouveau à élargir l'alignement en tenant
compte des erreurs de gap. Cette procédure est effectuée
par des méthodes de programmation dynamique et se repose
sur les réslultats fournis lors de l'étape précédente.
Ainsi, à partir des extrémités de l'alignement sans gap, les chercheurs
ont vérifié la possibilité d'inclure ou d'éliminer certains
caractères. Si le score de ce nouvel alignement est
supérieure à un seuil S2, alors l'étape 5 peut être
effectuée.

{\bf Etape 5:} visualistion des résultats. Durant cette dernière
étape, BLASTP fournie progressivement une visualisation
des alignements. Cettevisualisation élabore une méthode
de traceback sur la configuration de données pour bien
évaluer les emplacements des appariements entre
synmboles.

La rapidité de BLASTP en fait son principal avantage
par rapport aux techniques de programmation dynamique.
Cette rapidité fournie par BLASTP  est significative
particulièrement lorsque la taille W du point d'ancrage
est importante. En effet, plus W augmente, moins on
aura à trouver un mot de cette taille et donc moins les
étapes 3,4 et 5 seront exécutés. Ainsi, pour la
comparaison de protéines la taille est
conventionellement de 3 caractères.

Néanmoins, lorsque BLASTP est amener à traiter
d'important groupe de requête, l'algorithme doit
parcourir à chaque reprise la banque pour chaque
séquence requête.

\subsubsection{L'algorithme iBLASTP}

La banque et les requêtes sont tous les deux indexés
par l'algorithme de iBLASTP. La doublle indexation à ce
que il n'y est plus lieu de parcourir la banque.
En effet, à partir d'une structure de donnée
approprié, l'ensemble des mots identiques peuvent être
pointés à droite et à gauche des deux banques.

Par exemple, si un mot M est présent $n1$ fois dans la
banque n1 et $n2$ fois dans la banque n2, cela implique
qu'il y a $n1 \times n2$ points d'ancrage.
Donc, il y a dans ce cas $n1 \times n2$ calculs
d'alignements sans gap à calculer.
L'algorithme de iBLASTP s'effectue aussi en 5 étapes:

{\bf Algorithme de iBLASTP}

\begin{verbatim}
1: indexation des 2 banques sur la base                   - étape 1
       de mots de W caractères
2: pour chaque mot M différent
3:  construire une liste de n1 sous-séquence banque 1     - étape 2
4:   construire une liste de n2 sous-séquences banque 2
5: pour chaque sous-séquence du bloc 1
6:    pour chaque sous-séquence du bloc 2
7:       extension sans gap                               - étape 3
8:       Si score >= S1
9:          début d'extension avec gap                    - étape 4-1
10:        Si score >= S2
11:           extension complète avec gap                 - étape 4-2
12:        Si score >= S3
13:              afficher l'alignement                    - étape 5
\end{verbatim}

Etape 1: indexation. Le principe de point d'ancrage est gardé.
Cependant, contrairement à l'algorithme précédent,
l'indexation est effectuée sur les deux banques.
L'objectif de l'indexation est de stocker les
positions des mots similaires dans une liste.

Etape 2: recherche de points d'ancrage. La recherche
s'opère assez rapidement car il est question de
consulter les listes établies précédemment.
Une sous-séquence réfère à une instance du mot
entouré de son voisinage immédiat.

Etape 3: extension sans gap. Le calcul d'extension est
réalisé entre toutes les sous-séquences,
c'est-à-dire $n1 \times n2$ fois.

Etape 4: cette phase est divisée en deux sous étapes.
D'abord, l'espace de recherche de la programmation
dynamique est réduit.

Etape 5: visualisation des résultats.

\subsection{Parallélisation sur GPU}

La parallélisation sur GPU de la recherche de similaraités
de séquences a été réalisé sur une carte graphique NVIDIA
GeForce 8800 GTX. Dans cette carte, on trouve un GPU ayant
8 multiprocesseurs SIMD. Chacun d'eux possède 16
processueurs. La programmation CUDA a été exploitée.
La structure de la programmation prend enn compte une
grille composée d'un ensemble de blocs qui exécutent
en même temps maximum 256 threads.

\subsubsection{Implémentation de iBASTP sur GPU}

L'étape 3 de l'algorithme iBLASTP, extension sans gap,
génère un nombre important de calculs d'élargissement
sans gap $n1 \times n2$ pour un mot donné. Pour optimiser
l'algorithme de iBLASTP, l'équipe de développement a
eu l'idée de stoker les résultats produits par l'étape 3
et de réaliser l'étape 4-1 quand un nombre suffisant
d'extension san gap a été trouvé.

{\bf Algorithme de iBLASTP-GPU}

\begin{verbatim}
1 : indexation des 2 banques sur la base de mots de W caractères  – étape 1
2 : pour chaque mot M différent
3 :     construire une liste de n1 sous-séquences banque 1        – étape 2
4 :     construire une liste de n2 sous-séquences banque 2
5 :     effectuer n1 x n2 extensions sans gap sur GPU             – étape 3
6 :     stocker les extensions avec un score >= S1 dans R
7 :     si la liste R contient au moins K éléments
8 :         effectuer K début d’extensions avec gap sur GPU     – étape 4-1
9 :         si score >= S2
10:             extension complète avec gap                     – étape 4-2
11:         si score >= S3
12:             afficher l’alignement                             – étape 5
13:      Supprimer K éléments de R
\end{verbatim}

D'après l'équipe qui a mené les tests, les résultats de
l'étape 3 sont mémorisés dans la liste R. Les étapes
suivantes d'exécutent uniquement lorsque la liste R
a un nombre sufisant de résultats pour exploiter
correctement le GPU.

\subsubsection{Parallélisation de l'extension sans gap}

Dans l'algorithme, il y a $n1 \times n2$ extensions sans gap qui
doit être réalisé. Pour dimininuer les coûts d'échange
entre l'hôte et la carte graphique, l'équipe de chercheurs
a construit deux blocs de sous-séqunce: le bloc1[N,n1],
avec N la longueur des sous-séquences et le bloc2[n2,N].
Par la suite, les deux blocs sont envoyés à la carte
graphique pour réaliser le calcul des extensions sans gap.
Ainsi, un bloc C[n1,n2] va stocker les scores résultants.
D'après l'équipe qui a mené les tests, la valeur de chaque
cellule [i,j] du bloc C est le score d'extension sans gap
entre la sous-séquence j du bloc 1 (référant à la ligne i)
et la sous-séquence i du bloc 2 (corrspondant à la
colonne j).

\subsubsection{Parallélisation de l'extension avec gap}

Chaque K alignements avec gap va impliquer la
parallélisation de 2048 tâches sur le GPU. Pour éviter
tout conflit mémoire, l'équipe de développement a fait en
sorte qu'en début de calcul, chaque thread charge sa paire
de sous-séquence dans sa mémoire locale.
